\documentclass[10pt,a4paper,ragged2e,withhyper]{altacv}

\geometry{left=1.25cm,right=1.25cm,top=1.5cm,bottom=1.5cm,columnsep=1.2cm}

\usepackage{paracol}

\iftutex 
  % If using xelatex or lualatex:
  \setmainfont{Roboto Slab}
  \setsansfont{Lato}
  \renewcommand{\familydefault}{\sfdefault}
\else
  % If using pdflatex:
  \usepackage[rm]{roboto}
  \usepackage[defaultsans]{lato}
  % \usepackage{sourcesanspro}
  \renewcommand{\familydefault}{\sfdefault}
\fi

\definecolor{SlateGrey}{HTML}{2E2E2E}
\definecolor{LightGrey}{HTML}{4e4e4e} %666666
\definecolor{DarkPastelRed}{HTML}{0877d2} %450808 1354a4
\definecolor{PastelRed}{HTML}{1354a4} %8F0D0D  12257c
\definecolor{GoldenEarth}{HTML}{E7D192} 
\colorlet{name}{black}
\colorlet{tagline}{PastelRed}
\colorlet{heading}{DarkPastelRed}
\colorlet{headingrule}{GoldenEarth}
\colorlet{subheading}{PastelRed}
\colorlet{accent}{PastelRed}
\colorlet{emphasis}{SlateGrey}
\colorlet{body}{LightGrey}

\renewcommand{\namefont}{\Huge\rmfamily\bfseries}
\renewcommand{\personalinfofont}{\footnotesize}
\renewcommand{\cvsectionfont}{\LARGE\rmfamily\bfseries}
\renewcommand{\cvsubsectionfont}{\large\bfseries}


\renewcommand{\cvItemMarker}{{\small\textbullet}}
\renewcommand{\cvRatingMarker}{\faCircle}

\input{pubs-num.tex}


\addbibresource{sample.bib}

\begin{document}
\name{Théo HUET}
\tagline{Étudiant en Master de Physique Théorique}

\photoR{3.3cm}{photoCV}

\personalinfo{
  \email{theo6huet@gmail.com}
  \phone{+33 7 69 66 28 13}
  \location{Yvelines, France}
  \\
  \linkedin{theo-huet} \qquad \qquad \qquad
  \github{TheoHUETQC}

}

\makecvheader

\columnratio{0.6}

\begin{paracol}{2}
\cvsection{EXPÉRIENCES}

\cvevent{Stage de recherche de fin de Licence}{Laboratoire de Physique Théorique et Modélisation (LPTM)\\
CY Cergy Paris Université et CNRS}{Mai 2024}{Cergy-Pontoise (95)}
\begin{itemize}
    \item Étude de la mesure quantique, application à l’effet Zénon Quantique avec Monsieur Fratino.
    \item https://github.com/TheoHuetQC/intership-report-zenon-effect
\end{itemize}

\divider

\cvevent{Saisonnier, Restauration du Stade Léo Lagrange}{Mairie Des Mureaux}{été 2021}{Les Mureaux (78)}
\begin{itemize}
    \item Peinture, ponçage, réalisation de photo et d'un diaporama.
\end{itemize}

\medskip

\cvsection{FORMATIONS}
\cvevent{Master Physique option Physique Théorique}{CY Cergy Paris Université -- Cergy-Pontoise (95)}{2024 -- Maintenant}{}
option Informatique Quantique

\divider

\cvevent{Double Licence mention Mathématiques et Physique}{CY Cergy Paris Université -- Cergy-Pontoise (95)}{2021 -- 2024}{}
Mention assez bien \\
1ère et 2ème année en Cycle Universitaire Préparatoire aux Grandes Ecoles d’ingénieurs (CUPGE - MP)

\divider

\cvevent{Baccalauréat général spécialité Mathématiques -- Physique}{Lycée agricole et horticole -- Saint-Germain-en-Laye (78)}{2021}{}
Option Maths experts, Mention assez bien

\medskip

\cvsection{Projets} %projet generateur de mot de passe et projet informatique avec kylian

\cvevent{Générateur de mot de passe}{https://github.com/TheoHuetQC/passworld-generator}{}{}
\begin{itemize}
\item Application developpé avec Python, Tkinter et hashlib.
\item Découverte et utilisation d'outils graphique et du hachage.
\end{itemize}

\switchcolumn

\cvsection{COMPÉTENCES}

Langages informatique :\\ \medskip
\cvskill{Python}{4.5}
\medskip
\cvskill{\textit{C}}{1}
\medskip
\cvskill{\LaTeX}{3} 

\divider

Bureautique :
\\ \medskip
\cvskill{Microsoft Office}{5}
\medskip
\cvskill{Google Workspace}{4}
\medskip
\cvskill{Da Vinci Resolve}{3} 
\medskip
\cvskill{Photoshop}{3}
\medskip
\cvskill{Modélisation 3D}{2}

\medskip

\cvsection{QUALITÉS}

{\LaTeXraggedright
\cvtag{Patience}
\cvtag{Curiosité}
\cvtag{Autodidacte} %Autonomie
\cvtag{Créativité} %\cvtag{créativité \& réactivité}
\par}

\medskip

\cvsection{Langues}

\begin{table}[h]
\begin{tabular}{l@{\hskip 0.8in}l@{\hskip 0.5in}l}
Français   &   & Maternelle  \\
\multicolumn{3}{l}{\divider} \\
Anglais    &  \quad & Niveau B2          \\
\multicolumn{3}{l}{\divider} \\
Espagnol   &  \quad & Niveau A2          \\
\multicolumn{3}{l}{\divider} \\
Russe      &  \quad & Niveau A1         
\end{tabular}
\end{table}

\cvsection{Loisirs}

\begin{table}[h]
\begin{tabular}{l}
Développement en Python \\
\divider                \\
Dessin                  \\
\divider                \\
Jeux vidéos             \\
\divider                \\
Bricolage               \\
\divider                \\
Vélo                   
\end{tabular}
\end{table}

\end{paracol}


\end{document}