%%%%%%%%%%%%%%%%%
% This is an sample CV template created using altacv.cls
% (v1.7.2, 28 August 2024) written by LianTze Lim (liantze@gmail.com). Compiles with pdfLaTeX, XeLaTeX and LuaLaTeX.
%
%% It may be distributed and/or modified under the
%% conditions of the LaTeX Project Public License, either version 1.3
%% of this license or (at your option) any later version.
%% The latest version of this license is in
%%    http://www.latex-project.org/lppl.txt
%% and version 1.3 or later is part of all distributions of LaTeX
%% version 2003/12/01 or later.
%%%%%%%%%%%%%%%%

%% Use the "normalphoto" option if you want a normal photo instead of cropped to a circle
% \documentclass[10pt,a4paper,normalphoto]{altacv}

\documentclass[10pt,a4paper,ragged2e,withhyper]{altacv}
%% AltaCV uses the fontawesome5 and simpleicons packages.
%% See http://texdoc.net/pkg/fontawesome5 and http://texdoc.net/pkg/simpleicons for full list of symbols.

% Change the page layout if you need to
\geometry{left=1.25cm,right=1.25cm,top=1.5cm,bottom=1.5cm,columnsep=1.2cm}

% The paracol package lets you typeset columns of text in parallel
\usepackage{paracol}

% Change the font if you want to, depending on whether
% you're using pdflatex or xelatex/lualatex
% WHEN COMPILING WITH XELATEX PLEASE USE
% xelatex -shell-escape -output-driver="xdvipdfmx -z 0" sample.tex
\iftutex 
  % If using xelatex or lualatex:
  \setmainfont{Roboto Slab}
  \setsansfont{Lato}
  \renewcommand{\familydefault}{\sfdefault}
\else
  % If using pdflatex:
  \usepackage[rm]{roboto}
  \usepackage[defaultsans]{lato}
  % \usepackage{sourcesanspro}
  \renewcommand{\familydefault}{\sfdefault}
\fi

% Change the colours if you want to
\definecolor{SlateGrey}{HTML}{2E2E2E}
\definecolor{LightGrey}{HTML}{4e4e4e} %666666
\definecolor{DarkPastelRed}{HTML}{0877d2} %450808 1354a4
\definecolor{PastelRed}{HTML}{1354a4} %8F0D0D  12257c
\definecolor{GoldenEarth}{HTML}{E7D192} 
\colorlet{name}{black}
\colorlet{tagline}{PastelRed}
\colorlet{heading}{DarkPastelRed}
\colorlet{headingrule}{GoldenEarth}
\colorlet{subheading}{PastelRed}
\colorlet{accent}{PastelRed}
\colorlet{emphasis}{SlateGrey}
\colorlet{body}{LightGrey}

% Change some fonts, if necessary
\renewcommand{\namefont}{\Huge\rmfamily\bfseries}
\renewcommand{\personalinfofont}{\footnotesize}
\renewcommand{\cvsectionfont}{\LARGE\rmfamily\bfseries}
\renewcommand{\cvsubsectionfont}{\large\bfseries}


% Change the bullets for itemize and rating marker
% for \cvskill if you want to
\renewcommand{\cvItemMarker}{{\small\textbullet}}
\renewcommand{\cvRatingMarker}{\faCircle}
% ...and the markers for the date/location for \cvevent
% \renewcommand{\cvDateMarker}{\faCalendar*[regular]}
% \renewcommand{\cvLocationMarker}{\faMapMarker*}


% If your CV/résumé is in a language other than English,
% then you probably want to change these so that when you
% copy-paste from the PDF or run pdftotext, the location
% and date marker icons for \cvevent will paste as correct
% translations. For example Spanish:
% \renewcommand{\locationname}{Ubicación}
% \renewcommand{\datename}{Fecha}


%% Use (and optionally edit if necessary) this .tex if you
%% want to use an author-year reference style like APA(6)
%% for your publication list
% \input{pubs-authoryear.tex}

%% Use (and optionally edit if necessary) this .tex if you
%% want an originally numerical reference style like IEEE
%% for your publication list
\input{pubs-num.tex}

%% sample.bib contains your publications
\addbibresource{sample.bib}
% \usepackage{academicons}\let\faOrcid\aiOrcid
\begin{document}
\name{Théo HUET}
\tagline{Étudiant en Master de Physique Théorique}
%% You can add multiple photos on the left or right
\photoR{3.3cm}{photoCV}
% \photoL{2.5cm}{Yacht_High,Suitcase_High}

\personalinfo{%
  % Not all of these are required!
  \email{theo6huet@gmail.com}
  \phone{+33 7 69 66 28 13}
  %\mailaddress{5 Parc Delage Les Mureaux 78130}
  \location{Yvelines, France}
  %\homepage{www.homepage.com}
  % \twitter{@twitterhandle}
  %\xtwitter{@x-handle}
  \\
  \linkedin{théo-huet-9a2bb0347}
  \github{TheoHUETQC}
  %\orcid{0000-0000-0000-0000}
  
  %% You can add your own arbitrary detail with
  %% \printinfo{symbol}{detail}[optional hyperlink prefix]
  % \printinfo{\faPaw}{Hey ho!}[https://example.com/]

  %% Or you can declare your own field with
  %% \NewInfoFiled{fieldname}{symbol}[optional hyperlink prefix] and use it:
  % \NewInfoField{gitlab}{\faGitlab}[https://gitlab.com/]
  % \gitlab{your_id}
  %%
  %% For services and platforms like Mastodon where there isn't a
  %% straightforward relation between the user ID/nickname and the hyperlink,
  %% you can use \printinfo directly e.g.
  % \printinfo{\faMastodon}{@username@instace}[https://instance.url/@username]
  %% But if you absolutely want to create new dedicated info fields for
  %% such platforms, then use \NewInfoField* with a star:
  % \NewInfoField*{mastodon}{\faMastodon}
  %% then you can use \mastodon, with TWO arguments where the 2nd argument is
  %% the full hyperlink.
  % \mastodon{@username@instance}{https://instance.url/@username}
}

\makecvheader
%% Depending on your tastes, you may want to make fonts of itemize environments slightly smaller
% \AtBeginEnvironment{itemize}{\small}

%% Set the left/right column width ratio to 6:4.
\columnratio{0.6}

% Start a 2-column paracol. Both the left and right columns will automatically
% break across pages if things get too long.
\begin{paracol}{2}
\cvsection{EXPÉRIENCES}

\cvevent{Stage de recherche de fin de Licence}{Laboratoire de Physique Théorique et Modélisation (LPTM)\\
CY Cergy Paris Université et CNRS}{Mai 2024}{Cergy-Pontoise (95)}
\begin{itemize}
    \item Étude de la mesure quantique, application à l’effet Zénon Quantique avec Monsieur Fratino.
    \item https://github.com/TheoHuetQC/intership-report-zenon-effect
\end{itemize}

\divider

\cvevent{Saisonnier, Restauration du Stade Léo Lagrange}{Mairie Des Mureaux}{été 2021}{Les Mureaux (78)}
\begin{itemize}
    \item Peinture, ponçage, réalisation de photo et d'un diaporama.
\end{itemize}

\medskip

\cvsection{FORMATION}
\cvevent{Master Physique option Physique Théorique}{CY Cergy Paris Université -- Cergy-Pontoise (95)}{2024 -- Maintenant}{}
option Informatique Quantique

\divider

\cvevent{Double Licence mention Mathématiques et Physique}{CY Cergy Paris Université -- Cergy-Pontoise (95)}{2021 -- 2024}{}
Mention assez bien \\
1ère et 2ème année en Cycle Universitaire Préparatoire aux Grandes Ecoles d’ingénieurs (CUPGE - MP)

\divider

\cvevent{Baccalauréat général spécialité Mathématiques -- Physique}{Lycée agricole et horticole -- Saint-Germain-en-Laye (78)}{2021}{}
Option Maths experts, Mention assez bien

\medskip

\cvsection{Projets} %projet generateur de mot de passe et projet informatique avec kylian

\cvevent{Générateur de mot de passe}{https://github.com/TheoHuetQC/passworld-generator}{}{}
\begin{itemize}
\item Application developpé avec Python, Tkinter et du hachage.
\end{itemize}

% Adapted from @Jake's answer from http://tex.stackexchange.com/a/82729/226
% \wheelchart{outer radius}{inner radius}{
% comma-separated list of value/text width/color/detail}
%\wheelchart{1.5cm}{0.5cm}{%
%  6/8em/accent!30/{Sleep,\\beautiful sleep},
%  3/8em/accent!40/Hopeful novelist by night,
%  8/8em/accent!60/Daytime job,
%  2/10em/accent/Sports and relaxation,
%  5/6em/accent!20/Spending time with family
%}

% use ONLY \newpage if you want to force a page break for
% ONLY the current column

%% Switch to the right column. This will now automatically move to the second
%% page if the content is too long.
\switchcolumn

\cvsection{COMPÉTENCES}

%\cvachievement{\faTrophy}{Fantastic Achievement}{and some details about it}
Langages informatique :\\ \medskip
\cvskill{Python}{4.5}
\medskip
\cvskill{\textit{C}}{1}
\medskip
\cvskill{\LaTeX}{3} 

\divider

Bureautique :
\\ \medskip
\cvskill{Microsoft Office}{5}
\medskip
\cvskill{Google Workspace}{4}
\medskip
\cvskill{Da Vinci Resolve}{3} 
\medskip
\cvskill{Photoshop}{3}
\medskip
\cvskill{Modélisation 3D}{2}

\medskip

\cvsection{QUALITÉS}
%% Yeah I didn't spend too much time making all the
%% spacing consistent... sorry. Use \smallskip, \medskip,
%% \bigskip, \vspace etc to make adjustments.

% Don't overuse these \cvtag boxes — they're just eye-candies and not essential. If something doesn't fit on a single line, it probably works better as part of an itemized list (probably inlined itemized list), or just as a comma-separated list of strengths.

% The `ragged2e` document class option might cause automatic linebreaks between \cvtag to fail.
% Either remove the ragged2e option; or 
% add \LaTeXraggedright in the paragraph for these \cvtag
{\LaTeXraggedright
\cvtag{Patience}
\cvtag{Curiosité}
\cvtag{Autonomie}
\cvtag{Créativité} %\cvtag{créativité \& réactivité}
\par}

\medskip

\cvsection{Languages}

\begin{table}[h]
\begin{tabular}{l@{\hskip 0.8in}l@{\hskip 0.5in}l}
Français   &   & Maternelle  \\
\multicolumn{3}{l}{\divider} \\
Anglais    &  \quad & Niveau B2          \\
\multicolumn{3}{l}{\divider} \\
Espagnol   &  \quad & Niveau A2          \\
\multicolumn{3}{l}{\divider} \\
Russe      &  \quad & Niveau A1         
\end{tabular}
\end{table}

\cvsection{Loisirs}

\begin{table}[h]
\begin{tabular}{l}
Développement en Python \\
\divider                \\
Dessin                  \\
\divider                \\
Jeux vidéos             \\
\divider                \\
Bricolage               \\
\divider                \\
Vélo                   
\end{tabular}
\end{table}

\end{paracol}


\end{document}
